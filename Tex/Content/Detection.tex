\chapter{IMSI Catcher Detection}
\section{Framework and Hardware}
The following section will give a short overview of the OsmocomBB framework and how it works in conjunction with the Motorola C123 mobile phone to enable information harvesting for the \gls{icds}.
OsmocomBB is one of many \gls{osmo} projects\footnote{\url{http://osmocom.org/}} that implements the software part of a mobile phone.
Another project is OpenBSC which implements software for configuring and operating a \gls{bsc}.
OpenBSC is used to realise the Open Source IMSI Catcher \cite{dennis} and the base station that will be used later to evaluate the performance of the \gls{icds}.

\subsection{OsmocomBB}
OscmocomBB is the project that implements the baseband part of \gls{gsm} as an open source project.
The goal is to have, by using compatible hardware, a phone using free software only as opposed proprietary baseband implementations.
This could be beneficial to multiple areas \cite{osmo_rationale}:
\begin{itemize}
	\item \textbf{Security:} The software running on the baseband chips is highly proprietary and closed.
	One cannot be sure that this software does not have bugs that could be exploited and ultimately pose a security risk to the subscriber.
	History has shown that open source projects are more secure than proprietary solutions since more people can view the source to find issues.
	If a security threat is found the bug is fixed fast and a patch is released.
	This could be a great benefit for phone users.
	\item \textbf{Education:} Currently knowledge about \gls{gsm} and its layers on a technical level is not very well spread.
	The literature so far 
	An open source implementation as a reference could serve to educate more developers generally interested in the subject of mobile communications and thus improve products and software.
	\item \textbf{Research:} An open source implementation can decouple research on \gls{gsm} technologies from the industry since key technologies are no longer only available to researchers employed to a specific company.	
\end{itemize}

The project targets \gls{gsm} layers 1-3 with the first layer being already implemented and ported to an open source firmware.
At this point layer two and three are do not actually run on the phone but rather on a computer to which the phone is connected via a serial cable.
More information on the compatible phones will be presented in Section \ref{sec:osmo_phones}.

\begin{figure}
\centering
\caption{Interaction of the OsmocomBB components with the ICDS software.}
\label{fig:osmo_setup}
\end{figure}

The setup that is used for the \gls{icds} project can be seen in Figure \ref{fig:osmo_setup}.
It was build and tested in a Xubuntu 11.10 environment \footnote{http://xubuntu.org/} which is a more lightweight variant of the popular Debian based Ubuntu Linux distribution.
The process of acquiring, compiling and running the OsmocomBB framework itself in this environment is explained in Appendix \ref{sec:osmo_install}.
As can be seen in the diagram, layer 1 of the OsmocomBB \gls{gsm} stack runs on the phone.
It is connected via a serial cable to the computer running the \gls{icds}.
On the computer side the \texttt{osmocon} program provides a general interface to the phone.
\texttt{Osmocon} is also used to download the firmware to the Motorola C123.
Other software can communicate with \texttt{osmocon} and subsequently with the phone using unix sockets.

\texttt{Catcher} is a modified version of the \texttt{cell\_log} program by Andreas Eversberg that interfaces with \texttt{osmocon} to harvest information from \gls{bts} and forward it to the \gls{icds}.
It can be seen as a layer 2/3 program that scans through available frequencies and reads information from the \gls{bcch} whenever one such channel is available on the frequency at hand.
The forwarding is done directly via \texttt{stdout} since it runs as a child process of the \gls{icds}.
The functionality of \texttt{catcher} will be explained in detail in Section \ref{sec:info_gathering} while the implementation and operation of the \gls{icds} will be discussed in Section \ref{sec:icds}.


\subsection{Motorola C123}
\label{sec:osmo_phones}

\section{Procedure}
\subsection{Information Gathering}
\label{sec:info_gathering}
\subsection{Information Evaluation}

\section{IMSI Catcher Detection System}
\label{sec:icds}
\subsection{Implemetation}
\subsubsection{Architecture}
\subsubsection{Extensions}
\subsection{Configuration}
\subsection{Operation}