\chapter{Introduciton}
Boundless communication for everyone, everywhere, anytime.
That was the main idea and dream behind the development of the \gls{gsm} technology.
Considering its reception and growth \cite{GSM2009,GSM_history2011,GSM_stats2011} it can be said that \gls{gsm} was one of the most successful technologies of the last 30 years.
Since the advent of portable radio equipment and portable microprocessors, mobile phones became technologically possible in the 80's.
From this point on, 

\section{Structure}
The remainder of this thesis is structured as follows:
Chapter \ref{ch:gsm} will give an overview of how the \gls{gsm} network is structured as well as describe the different components needed for operation and how they work together.
The second part of this chapter will discuss how the $U_m$ interface, or air interface works and what kind of information can be drawn off this interface.
The last part shows how an IMSI-Catcher works and where is it situated in the network shown before.
Possible attacks of how an IMSI-Catcher can be introduced in such a network are listed as well.
Finally there will be a discussion about the judicial situation in Germany concerning means of electronic surveillance for crime prevention and how this affects privacy and the basic rights of citizens.

The next chapter outlines the frameworks and the hardware that was used for this project.
